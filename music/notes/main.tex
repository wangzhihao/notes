\documentclass[UTF8]{ctexart}

\title{钢琴练琴笔记}

\begin{document}
\maketitle

\section{原则}

\begin{enumerate}
    \item \textbf{多思考},每次要给自己找一个课题,要知道自己是在练什么,有目的的练琴
    \item 每个问题不要占用太长时间,在多个问题间切换,因为长时间练同一个问题,效率会下降
    \item 注意那些细节,正是因为这些细节没有做好,所以弹不好听
    \item \textbf{慢速弹},不要着急合手,弹对才是最要紧的
\end{enumerate}

\section{节奏练习}

\begin{enumerate}
    \item 心里要有那个节奏点
    \item 十六分音符不要越弹越快,要匀称 
    \item 小幅点不要抢拍 
\end{enumerate}


\section{强弱练习}

\begin{enumerate} 
    \item 表现出乐句, 不要突然的强或者突然的弱,大部分这种时候都是没弹好
    \item 一个句子很多时候是渐强或者渐弱,不要在渐强的开始大拇指用非常大的力,也不要渐弱的结尾因为不好的习惯而弹重
    \item 要通过手在钢琴上的重力而左右方向控制强弱 
\end{enumerate}



\section{哈农练习}

\begin{enumerate}
    \item 高抬指,虽然演奏的时候不需要所有都高抬指,但是练习的时候需要。高抬指可以让手指更主动从而避免手腕和手臂过多的干预。
    \item 尤其是三指不要独立抬指。
    \item 慢速练习时,只是支撑的时候久一点。其他的动作还是一样快。
    \item 手臂和手腕要放松,手臂和手腕的戏不能多于手指。
\end{enumerate}

\section{和弦练习}

\begin{enumerate}
    \item 跳音的和弦要抓,再带着一点向下推的感觉。手指的动作幅度不用太大,幅度太大的动作有很多无用的动作。 
    \item 连音的和弦要放,但是手指仍然可以主动一点 。
\end{enumerate}

\section{演奏练习}

\begin{enumerate}
    \item 多听别人弹的
    \item 演奏的时候不必要抬高的地方就不要抬高,避免多余的动作。 
    \item 手臂和手腕要放松,手臂和手腕的戏不能多于手指。
    \item 手指的动作一定要有。
    \item 连线该断处要断开
    \item 在手指也有动作的时候,手腕也可以送一点力量来作出重心和强弱。
\end{enumerate}

\end{document}
