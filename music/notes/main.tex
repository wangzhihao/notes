\documentclass[11pt]{article}
\usepackage[utf8]{inputenc}
\usepackage{CJKutf8}
\usepackage{amsthm}
\usepackage{amssymb}
\usepackage{amsmath}
\usepackage{graphicx}
\usepackage{hyperref}
\usepackage{csquotes}
\usepackage{url}
\usepackage{multicol}


\title{钢琴练琴笔记}

\begin{document}
\maketitle

\section{原则}

\begin{enumerate}
    \item 每个问题不要占用太长时间,在多个问题间切换,因为长时间练同一个问题,效率会下降
    \item 每次要给自己找一个课题,要知道自己是在练什么,有目的的练琴
    \item 注意那些细节,正是因为这些细节没有做好,所以弹不好听
\end{enumerate}

\section{哈农练习}

\begin{enumerate}
    \item 高抬指,虽然演奏的时候不需要所有都高抬指,但是练习的时候需要。高抬指可以让手指更主动从而避免手腕和手臂过多的干预。
    \item 尤其是三指不要独立抬指。
    \item 慢速练习时,只是支撑的时候久一点。其他的动作还是一样快。
    \item 手臂和手腕要放松,手臂和手腕的戏不能多于手指。
\end{enumerate}

\section{和弦练习}

\begin{enumerate}
    \item 跳音的和弦要抓,再带着一点向下推的感觉。手指的动作幅度不用太大,幅度太大的动作有很多无用的动作。 
    \item 连音的和弦要放,但是手指仍然可以主动一点 。
\end{enumerate}

\section{演奏练习}

\begin{enumerate}
    \item 演奏的时候不必要抬高的地方就不要抬高,避免多余的动作。 
    \item 手臂和手腕要放松,手臂和手腕的戏不能多于手指。
    \item 手指的动作一定要有。
    \item 连线该断处要断开
    \item 在手指也有动作的时候,手腕也可以送一点力量来作出重心和强弱。
\end{enumerate}

\end{document}
