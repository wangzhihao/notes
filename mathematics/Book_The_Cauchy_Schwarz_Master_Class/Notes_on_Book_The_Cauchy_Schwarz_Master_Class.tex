\documentclass{article}
\usepackage{amsthm}
\usepackage{amssymb}
\usepackage{amsmath}
\usepackage{cleveref}


\newtheorem{theorem}{Theorem}[section]
\newtheorem{corollary}{Corollary}[theorem]
\newtheorem{proposition}{Proposition}[section]
\newtheorem{lemma}[theorem]{Lemma}
\newtheorem{problem}{Problem}[section]
\newtheorem{exercise}{Exercise}[section]

\title{Notes on Book The Cauchy Schwarz Master Class}
\author{Zhihao hao@chopin.fm}
\date{\today}

\begin{document}

\maketitle
\section{Chapter 1}

\begin{problem}
    Prove Cauchy's inequality.

    \begin{equation}
        \sum_i^n{a_ib_i} \leq \sqrt{\sum_i^n{a_i^2}}  \sqrt{\sum_i^n{b_i^2}}
    \end{equation}

    Moreover, if you already know a proof of Cauchy's inequality, find another one!

\end{problem}

\begin{proof}
    We can prove it in algebra way.

    Firstly, if the square of the inequality holds, the inequality will also hold. Since the right side is non-negative. From this view, we can transform
    the problem to prove the square of the inequality as follows.

    \begin{equation*}
        (\sum_i^n{a_ib_i})^2 \leq \sum_i^n{a_i^2}  \sum_i^n{b_i^2}
    \end{equation*}

    Let's define $\Delta$ as the subtraction of the left side from the right side,

    \begin{equation*}
        \Delta = \sum_i^n{a_i^2}  \sum_i^n{b_i^2} - (\sum_i^n{a_ib_i})^2
    \end{equation*}

    which can be reduced to

    \begin{align*}
        \Delta & = \sum_{i,j}{a_i^{2}b_j^{2}} - \sum_{i,j}{a_ib_ia_jb_j} \\
               & = \sum_{i,j}{a_ib_j(a_ib_j - a_jb_i)}
    \end{align*}

    Now if we add $\Delta$ with itself, since the indices $i,j$ are interchangable, we will end up with the following:

    \begin{align*}
        \Delta + \Delta & = \sum_{i,j}{a_ib_j(a_ib_j - a_jb_i)} + \sum_{i,j}{a_jb_i(a_jb_i - a_ib_j)} \\
               & = \sum_{i,j}{a_ib_j(a_ib_j - a_jb_i) - a_jb_i(a_ib_j - a_jb_i)} \\
               & = \sum_{i,j}{(a_ib_j - a_jb_i)^2} \\
               & \geq 0
    \end{align*}

    We can also prove from the intuitive geometry or vector way, which is omitted here.
\end{proof}

\begin{exercise}
    For Book Exercise 1.1. Prove for each real sequence $a_1, a_2, ..., a_n$ one has:

    \begin{equation*}
        a_1 + a_2 + ... + a_n \leq \sqrt{n}(a_1^2 + a_2^2 + ... + a_n^2)^{\frac{1}{2}}
    \end{equation*}

\end{exercise}

\begin{proof}
    We can view the sum of the sequence as the inner product of vectors in $\mathbb{R}_n$: $(a_1, a_2, ..., a_n)$ and $(1, 1, ..., 1)$. Then it becomes obvious from Cauchy's inequality.
\end{proof}
\begin{exercise}
    For Book Exercise 1.1. Prove

    \begin{equation*}
        \sum_{k=1}^{n}{a_k} \leq (\sum_{k=1}^{n}{|a_k|^{2/3}})^{\frac{1}{2}} (\sum_{k=1}^{n}{|a_k|^{4/3}})^{\frac{1}{2}}
    \end{equation*}

\end{exercise}

\begin{proof}
    We can view the sum as the inner product of vectors in $\mathbb{R}_n$: $(a_1^{\frac{1}{3}}, a_2^{\frac{1}{3}}, ..., a_n^{\frac{1}{3}})$ and $(a_1^{\frac{2}{3}}, a_2^{\frac{2}{3}}, ..., a_n^{\frac{2}{3}})$. Then it becomes obvious from Cauchy's inequality.
\end{proof}

\begin{exercise}
    For Book Exercise 1.2. Prove

    \begin{equation*}
        1 \leq \bigg\{\sum_{j = 1}^{n}{p_ja_j}\bigg\} \bigg\{\sum_{j = 1}^{n}{p_jb_j}\bigg\}
    \end{equation*}

\end{exercise}

\begin{proof}
    We can expand the multipication.
    \begin{align*}
        & \bigg\{\sum_{j = 1}^{n}{p_ja_j}\bigg\} \bigg\{\sum_{j = 1}^{n}{p_jb_j}\bigg\} \\
        & = \sum_{i = 1}^{n}{p_i^2a_ib_i} + \sum_{i = 1}^{n}\sum_{j = i+1}^n{p_ip_j(a_ib_j + a_jb_i)} \\
        & \geq \sum_{i = 1}^{n}{p_i^2} + \sum_{i = 1}^{n}\sum_{j = i+1}^n{p_ip_j(a_ib_j + a_jb_i)}
    \end{align*}
    Since we know by the additive bound.
    \begin{equation*}
        a_ib_j + a_jb_i \geq 2(a_ib_j * a_jb_i)^{\frac{1}{2}} = 2(a_ib_i * a_jb_j)^{\frac{1}{2}} \geq 2(1*1)^{\frac{1}{2}} = 2
    \end{equation*}
    The equation would be
    \begin{align*}
        & \bigg\{\sum_{j = 1}^{n}{p_ja_j}\bigg\} \bigg\{\sum_{j = 1}^{n}{p_jb_j}\bigg\} \\
        & \geq \sum_{i = 1}^{n}{p_i^2} + \sum_{i = 1}^{n}\sum_{j = i+1}^n{p_ip_j(a_ib_j + a_jb_i)} \\
        & \geq \sum_{i = 1}^{n}{p_i^2} + \sum_{i = 1}^{n}\sum_{j = i+1}^n{p_ip_j*2} \\
        & = \bigg(\sum_{i = 1}^n{p_i}\bigg)^2 \\
        & = 1
    \end{align*}
\end{proof}

\begin{exercise}
    For Book Exercise 1.3 (Why Not Three or More?)
\end{exercise}

\begin{proof}
    We can apply the same trick(Cauchy's inequality) of Exercise 1 to solve Part (a).  Granted Part(a) is solved, we can solve Part(b) based on Part(a). We only need to prove the following.
    \begin{equation*}
        \sum_{k=1}^n{a_k^4} \leq \bigg( \sum_{k=1}^n{a_k^2} \bigg) ^2
    \end{equation*}
    We can just expand the polynomial on the RHS and only keep the $a_k^4$ terms, then we can easily get the inequality.

    Alternatively, We can also prove Part(b) and its more general form as follows, directly by Cauchy's inequality.
    \begin{equation*}
        \bigg( \sum_{k=1}^ns_{1_k}s_{2_k}...s_{m_k} \bigg) ^2
        \leq \sum_{k=1}^n{s_{1_k}^2} \sum_{k=1}^n{s_{2_k}^2} ... \sum_{k=1}^n{s_{m_k}^2}
    \end{equation*}
    Here is the proof.
    \begin{align*}
        & \bigg( \sum_{k=1}^ns_{1_k}s_{2_k}...s_{m_k} \bigg) ^2 \\
        & = \bigg( \sum_{k=1}^n(s_{1_k}s_{2_k}...s_{{m-1}_k})*(s_{m_k}) \bigg) ^2 \\
        & \leq \sum_{k=1}^n{(s_{1_k}s_{2_k}...s_{{m-1}_k})^2} \sum_{k=1}^n{s_{m_k}^2} & \text{by Cauchy's inequality}\\
        & \leq \bigg( \sum_{k=1}^n{s_{1_k}^2} \sum_{k=1}^n{s_{2_k}^2} ... \sum_{k=1}^n{s_{{m-1}_k}^2} \bigg) \sum_{k=1}^n{s_{m_k}^2} & \text{by polynomial expansion}\\
        & = \sum_{k=1}^n{s_{1_k}^2} \sum_{k=1}^n{s_{2_k}^2} ... \sum_{k=1}^n{s_{m_k}^2}
    \end{align*}
\end{proof}

\end{document}

