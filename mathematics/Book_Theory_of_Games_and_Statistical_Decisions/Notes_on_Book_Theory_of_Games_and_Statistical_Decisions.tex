\documentclass{article}
\usepackage{amsthm}
\usepackage{amsmath}
\usepackage{cleveref}


\newtheorem{theorem}{Theorem}[section]
\newtheorem{corollary}{Corollary}[theorem]
\newtheorem{proposition}{Proposition}[section]
\newtheorem{lemma}[theorem]{Lemma}

\title{Notes on Book Theory of Games and Statistical Decisions}
\author{Zhihao hao@chopin.fm}
\date{\today}

\begin{document}

\maketitle
\section{Chapter 1}
To read.

\section{Chapter 2}

\begin{lemma} \label{lem1}
For any set $R$, the set $R^{*}$ of all points $r^{*}$ which are centers of gravity of a finite set of points in $R$ with weights, i.e., which are all representable as

\begin{equation}
 r^{*} = \lambda_{1}r_{1} + ... + \lambda_{k}r_{k}
\end{equation}

where $\lambda_i \geq 0,\quad \sum_{1}^{k}\lambda_i = 1, \quad r_i \in R, \quad i = 1 , ..., k, \quad k = 1, 2, ...,$

is a convex set containing $R$.
\end{lemma}

\begin{proof}
    Suppose we have two points $a$ and $b$ in $R^*$. as long as we can prove any $\alpha{}a + (1 -\alpha)b$ also in $R^*$, then by definition $R^*$ will be a convex set.

    \begin{align*}
        a & = \lambda_{i1}r_{i1} + ... + \lambda_{ik}r_{ik} \\
        b & = \lambda_{j1}r_{j1} + ... + \lambda_{jk}r_{jk} \\
        \alpha{} & a + (1 -\alpha)b = \\
            & \alpha{}\lambda_{i1}r_{i1} + ... + \alpha{}\lambda_{ik}r_{ik} + (1 -\alpha)\lambda_{j1}r_{j1} + ... + (1 -\alpha)\lambda_{jk}r_{jk}
    \end{align*}
    From the equation, $\alpha{}a + (1 -\alpha)b$ is of the same representation of $r^{*}$, so it also belongs to $R^{*}$.

\end{proof}

\begin{proposition} \label{prop1}
    \cref{lem1} can also be rephased as $R^*$, which is all possible convex combinations of the subset of $R$, is a convex set.
\end{proposition}


\begin{lemma}
    $R^*$ is the convex hull of $R$, i.e., the minimum convex set which contains $R$.
\end{lemma}

\begin{proof}
    Since in \cref{lem1} we've already proved that $R^*$ is a convex set, what's left is to prove the $R^*$ is the mininum one. By \cref{prop1}, we only need to prove
    that for any convex set $C$, which contains $R$, will contain all possible convex conbinations of subject of $R$.  

    Let's prove by induction. 2-way convex combinations of $R$ is in $C$, which holds by convex set definition, i.e, any $\alpha{}a + (1 -\alpha)b$ will belongs to $C$ if $a$ and $b$ belongs to $R$. If k-way convex combinations of $R$ is in $C$, then k + 1 way convex combinations of $R$ is also in $C$.
    \begin{align*}
        \lambda_{1}r_{1} & + ... + \lambda_{k}r_{k} + \lambda_{k+1}r_{k+1} = \\
          = & \sum_1^k{\lambda_{i}}(\frac{\lambda_{1}r_{1} + ... + \lambda_{k}r_{k}}{\sum_1^k{\lambda_{i}}}) + \lambda_{k+1}r_{k+1} \\
          = & \sum_1^k{\lambda_{i}} r^{*} + \lambda_{k+1}r_{k+1}
    \end{align*}
\end{proof}

\end{document}

