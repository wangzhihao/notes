\documentclass[12pt]{article}
\usepackage{amsmath}
\usepackage{graphicx}
\usepackage{hyperref}
\usepackage[latin1]{inputenc}

\title{Stream system survey}
\author{Zhihao Wang}

\begin{document}
\maketitle

\section{Challenges}
The following questions serve as criteria to verify the proposal architecture.

\subsection{Do independent fields delay each other?}
Inventory Health team has a fat table which contains about 30 fields of data. Some fields are pair-wise independent like Inventory Age and Glance view. But one field delays will also block the other one.

\subsection{Is it human fault-tolerant?}
Suppose the whole system crash down, How to recover to a consistent state? How to be human fault-tolerant?
\begin{quote}
Yet there's a form of fault-tolerance that's much more important than machine fault-tolerance: human fault-tolerance. If there's any certainty in software development, it's that developers aren't perfect and bugs will inevitably reach production. Our data systems must be resilient to buggy programs that write bad data \cite{latexcompanion}
\end{quote}

\begin{thebibliography}{9}
    \bibitem{latexcompanion} Marz, Nathan, How to beat the CAP theorem \texttt{http://nathanmarz.com/blog/how-to-beat-the-cap-theorem.html}.
\end{thebibliography}


\end{document}
