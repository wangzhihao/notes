\documentclass[12pt]{article}
\usepackage{amsmath}
\usepackage{graphicx}
\usepackage{hyperref}
\usepackage[latin1]{inputenc}

\title{Stream system survey}
\author{Zhihao Wang}

\begin{document}
\maketitle

\section{Problem}
The following questions serve as criteria to verify the proposal architecture.  \cite{latexcompanion}

\subsection{Do independent fields delay each other?}
Inventory Health team has a fat table which contains about 30 fields of data. Some fields are pair-wise independent like Inventory Age and Glance view. But one field delays will also block the other one.

\subsection{Is it human fault tolerant?}
Suppose the whole system crash down, How to recover to a consistent state? How to be human fault tolerant?
\begin{quote}
Yet there's a form of fault-tolerance that's much more important than machine fault-tolerance: human fault-tolerance. If there's any certainty in software development, it's that developers aren't perfect and bugs will inevitably reach production. Our data systems must be resilient to buggy programs that write bad data. \cite{latexcompanion}
\end{quote}

\begin{thebibliography}{9}
\bibitem{latexcompanion}
Michel Goossens, Frank Mittelbach, and Alexander Samarin.
\textit{The \LaTeX\ Companion}.
Addison-Wesley, Reading, Massachusetts, 1993.

\bibitem{einstein}
Albert Einstein.
\textit{Zur Elektrodynamik bewegter K{\"o}rper}. (German)
[\textit{On the electrodynamics of moving bodies}].
Annalen der Physik, 322(10):891–921, 1905.

\bibitem{knuthwebsite}
Knuth: Computers and Typesetting,
\\\texttt{http://www-cs-faculty.stanford.edu/\~{}uno/abcde.html}
\end{thebibliography}

\end{document}
